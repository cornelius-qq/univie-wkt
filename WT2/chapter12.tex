\chapter*{12. Radon-Nikodym-Ableitungen}
\addcontentsline{toc}{chapter}{12. Radon-Nikodym-Ableitungen}
In diesem Kapitel sei $(\Omega,\mathcal{A})$ ein messbarer Raum und $\nu,\mu$ generische Ma\ss{}e auf diesem Raum.
\section*{Signierte Ma\ss{}e}
 \addcontentsline{toc}{section}{Signierte Ma\ss{}e}
 
 \paragraph{12.1. Definition:} Ein signiertes Ma\ss{} ist eine Abbildung $\varphi:\mathcal{A}\to\R$, die $\sigma$-additiv ist, i.e. f\"ur $A_n\in\mathcal{A},n\geq1$ disjunkt gilt
 $$\varphi\left(\bigcup_{n\geq1}A_n\right)=\sum_{n\geq1}\varphi(A_n)$$
 In unserem Kontext seien signierte Ma\ss{}e immer endlich und bilden $\emptyset$ auf $0$ ab.
 
 \paragraph{12.2. Beispiel:} Einige Beispiele f\"ur signierte Ma\ss{}e sind
 \begin{enumerate}[label=(\roman*)]
     \item $\phi(A):=\mu(A)$, wenn $\mu$ endlich ist.
     \item $\varphi(A):=\mu(A)-\nu(A)$, wenn $\mu,\nu$ endlich sind.
     \item $\varphi(A):=\int_A f\ d\mu$, wenn $f\in L^1(\mu)$
 \end{enumerate}
 
 \paragraph{12.3. Lemma:} Seien $A_n\in\mathcal{A},n\geq1$, sodass entweder
 \begin{enumerate}[label=(\roman*)]
     \item $A_1\subseteq A_2\subseteq\hdots\subseteq\bigcup_{n\geq1}A_n:=A$, oder
     \item $A_1\supseteq A_2\supseteq\hdots\supseteq\bigcap_{n\geq1}A_n:=A$
 \end{enumerate}
 Dann ist in beiden F\"allen 
 $$\lim_{n\to\infty}\varphi(A_n)=\varphi(A)$$
 i.e. signierte Ma\ss{}e sind von unten/oben stetig.
 
 \paragraph{Beweis:}
 \begin{enumerate}[label=(\roman*)]
     \item
     \begin{align*}
     \varphi(A)&=\varphi(A_1\cup\bigcup_{n\geq1}(A_{n+1}\setminus A_n))\\
     &=\varphi(A_1)+\sum_{n\geq1}\varphi(A_{n+1}\setminus A_n)\\
     &=\lim_{N\to\infty}\left[\varphi(A_1)+\sum_{n=1}^N\varphi(A_{n+1}\setminus A_n)\right]\\
     &=\lim_{N\to\infty}\varphi(A_{N+1})=\lim_{N\to\infty}\varphi(A_N)
     \end{align*}
     \item Hier gilt $A_1^c\subseteq\hdots\subseteq\bigcup_{n\geq1}A_n^c=:A^c$
     \begin{flalign*}
         \overset{\text{(i)}}{\implies}&\varphi(A^c)=\lim_{n\to\infty}\varphi(A_n^c) &&\\
         \implies &\varphi(\Omega)-\varphi(A)=\lim_{n\to\infty}[\varphi(\Omega)-\varphi(A_n)] =\varphi(\Omega)-\lim_{n\to\infty}\varphi(A_n) &&\\
         \implies &\varphi(A)=\lim_{n\to\infty}\varphi(A_n)
     \end{flalign*}
     \end{enumerate}
     \qed
     
     \paragraph{4. Satz (Hahn/Jordan Decomposition Theorem):} Sei $\varphi$ ein signiertes Ma\ss{} auf $(\Omega,\mathcal{A})$. Dann gibt es eine Partition (Hahn-Zerlegung) von $\Omega$
     $$\Omega=\Omega_+\cup\Omega_-$$
     mit $\Omega_+,\Omega_-\in\mathcal{A}$, sodass
     $$\forall A\in\mathcal{A}:\varphi(A\cap\Omega_+)=:\varphi_+(A)\geq0, \ \varphi(A\cap\Omega_-)=:\varphi_-(A)\leq0$$
     Daraus folgt sofort die Jordan-Zerlegung
     $$\forall A\in\mathcal{A}:\varphi(A)=\varphi_+(A)+\varphi_-(A)$$
     
     \paragraph{Beweis:}Sei $\alpha:=\displaystyle\sup_{A\in\mathcal{A}}\varphi(A)$. Dann gibt es eine Menge $D\in\mathcal{A}$, sodass dieses Supremum angenommen wird, i.e. $\varphi(D)=\alpha$ (Details zur Konstruktion siehe unten).
     Setze nun $\Omega_+:=D$ und $\Omega_-:=D^c$. Dann gilt
     \begin{itemize}
         \item Angenommen es gibt $A\in\mathcal{A}$ mit $\varphi(A\cap D)<0$. Dann ist $D=(D\cap A)\cup(D\setminus A)$ und 
         $$\alpha=\varphi(D)=\varphi(D\cap A)+\varphi(D\setminus A)<\varphi(D\setminus A)$$
         ein Widerspruch, da $D\setminus A\in\mathcal{A}$ und $\alpha=\displaystyle\sup_{A\in\mathcal{A}}\varphi(A)$.
         \item Angenommen es gibt $A\in\mathcal{A}$ mit $\varphi(A\cap D^c)>0$. Dann ist 
         $$\varphi(D\cup(A\cap D^c))=\varphi(D)+\varphi(A\cap D^c)>\varphi(D)=/\alpha$$
         ein Widerspruch, da $D\cup(A\cap D^c)\in\mathcal{A}$ und $\alpha=\displaystyle\sup_{A\in\mathcal{A}}\varphi(A)$.
     \end{itemize}
     Zur Konstruktion von $D$:\newline
     W\"ahle $A_n,n\geq1$, sodass $\displaystyle\lim_{n\to\infty}\varphi(A_n)=\alpha$ und definiere f\"ur $n\geq1$
     $$\mathcal{B}_n:=\left\{B_I=\left(\bigcap_{I_i=1}A_i\right)\cup\left(\bigcap_{I_i=0}A_i\right):I\in\{0,1\}^n\right\}$$
     sukzessive feiner werdende Partitionen von $\Omega$. Setze nun 
     $$C_n:=\bigcup_{\substack{B_I\in\mathcal{B}_n\\ \varphi(B_I)>0}}B_I$$
     Nun ist 
     $$A_n=\bigcup_{\substack{B_I\in\mathcal{B}_n\\ I_n=1}}B_I$$
     sodass $\varphi(A_n)\leq\varphi(C_n)\ (*)$. Au\ss{}erdem ist 
     $$\varphi\left(\bigcup_{m=n}^NC_m\right)\leq\varphi\left(\bigcup_{m=n}^{N+1}C_m\right)$$
     da $\displaystyle\left(\bigcup_{m=n}^{N+1}C_m\right)\setminus \left(\bigcup_{m=n}^NC_m\right)$ eine Vereinigung von Mengen $B_I$ aus $\mathcal{B}_{N+1}$ mit $\varphi(B_I)>0$ ist. $(**)$\newline\newline 
     Setze nun $D_i:=\displaystyle\limsup_{n\to\infty}C_n$
     Dann folgt
     \begin{align*}
         \alpha&=\lim_{n\to\infty}\varphi(A_n) \\
         &\overset{(*)}{\leq}\liminf_{n\to\infty}\varphi(C_n)\\
         &\overset{(**)}{\leq}\liminf_{n\to\infty}\limsup_{N\to\infty}\varphi\left(\bigcup_{m=n}^NC_m\right)\\
         &\overset{\text{S.v.U.}}{=}\liminf_{n\to\infty}\varphi\left(\bigcup_{m=n}^\infty C_m\right)\\
         &\overset{\text{S.v.O.}}{=}\varphi\left(\bigcap_{n=1}^\infty \bigcup_{m=n}^\infty C_m\right)\\
         &=\varphi\left(\limsup_{n\to\infty}C_n\right)=\varphi(D)\leq\alpha
     \end{align*}
     \qed
     
     \paragraph{12.5. Definition:} 
     \begin{enumerate}[label=(\roman*)]
         \item $\mu$ und $\nu$ sind gegenseitig singular (kurz $\mu\perp\nu$), wenn
         $$\exists M\in\mathcal{A}:\mu(M)=0, \nu(M^c)=0$$
         Die Relation $\perp$ ist symmetrisch, i.e. $\mu\perp\nu\iff\nu\perp\mu$.
         \item $\nu$ ist absolut stetig bez\"uglich $\mu$, (kurz $\nu\ll\mu$) wenn
         $$\forall A\in\mathcal{A}: \mu(A)=0\implies\nu(A)=0$$
         Die Relation $\ll$ ist transitiv, i.e. $\nu\ll\mu,\mu\ll\lambda\implies\nu\ll\lambda$.
     \end{enumerate}
     
     \paragraph{12.6. Lemma:} Seien $\mu$ und $\nu$ zwei endliche Ma\ss{}e, die NICHT gegenseitig singular sind. Dann gibt es ein $\eps>0$ und $N\in\mathcal{A}$ mit $\mu(B)>0$, sodass
     $$\forall A\in\mathcal{A}:\eps\cdot\mu(A\cap B)\leq\nu(A\cap B)$$
     
     \paragraph{Beweis: }F\"ur $n\geq1$ definiere ein signiertes Ma\ss{} $\varphi_n:=\nu-\mu/n$. Seien nun $\Omega^{(n)}_+$ und $\Omega^{(n)}_-$ die Hahn-Zerlegung von $\varphi_n$. Setze $M:=\bigcup_{n\geq1}\Omega^{(n)}_+$ und $M^c=\bigcap_{n\geq1}\Omega^{(n)}_-$. Da $\varphi_1\leq\hdots\leq\displaystyle\lim_{n\to\infty}\varphi_n=\nu$, gilt $\Omega^{(1)}_+\supseteq\hdots\supseteq M^c$ und mit der S.v.O. f\"ur signierte Ma\ss{}e $\forall n\geq1:\varphi_n(M^c)\leq0$. \newline Damit folgtper Definition f\"ur alle $n\geq1$
     \begin{align*}
         \nu(M^c)-\dfrac{1}{n}\mu(M^c)\leq0 \iff \nu(M^c)\leq\dfrac{1}{n}\mu(M^c)
     \end{align*}
     und damit $\nu(M^c)=0$. Laut Annahme git aber $\nu\xcancel{\perp}\mu$ und damit $\mu(M)>0$. Also folgt
     $$0<\mu(M)\leq\sum_{n\geq1}\mu(\Omega^{(n)}_+)$$
     und damit gibt es ein $m\geq1$, sodass $\mu(\Omega^{(m)}_+)>0\ (*)$. Mit der Definition von $\varphi_m$ gilt aber
     $$\forall A\in\mathcal{A}:\nu(A\cap \Omega^{(m)}_+)-\dfrac{1}{m}\mu(A\cap \Omega^{(m)}_+)\geq0\iff\nu(A\cap \Omega^{(m)}_+)\geq\dfrac{1}{m}\mu(A\cap \Omega^{(m)}_+)$$
     Setze also $B:=\Omega^{(m)}_+$ mit $\eps:=m^{-1}$. Dann gilt wegen $(*)$ auch $\mu(B)=\mu(\Omega^{(m)}_+)>0$. \qed
     
    
     
     \paragraph{12.7. Satz (Radon\textendash Nikodym Theorem f\"ur $\sigma$-endliche Ma\ss{}e):} Seien $\mu$ und $\nu$ zwei $\sigma$-endliche Ma\ss{}e, sodass $\nu\ll\mu$. Dann hat $\nu$ eine Dichte bez\"uglich $\mu$, i.e. es gibt ein $f:\Omega\to\R$ messbar, sodass
     \begin{equation}
         \forall A\in\mathcal{A}: \nu(A)=\int_A f \ d\mu
     \end{equation}
     Wir schreiben $f=\dfrac{d\nu}{d\mu}$. Falls insbesondere $g:\Omega\to\R$ eine weitere messbare Abbildung ist, die (7) erf\"ullt, dann gilt $f=g$ $\mu$-a.e.
          
     \paragraph{Beweis:} Unterscheide hier die F\"alle, wo $\mu,\nu$ beide endlich sind, und den allgemeinen Fall ($\mu,\nu$ beide $\sigma$-endlich).\newline\newline
     1. Fall ($\mu,\nu$ beide endlich): Betrachte die Menge
     $$\mathcal{G}:=\left\{g\geq0:g \text{ messbar und } \forall A\in\mathcal{A}:\int_A g\ d\mu\leq\nu(A)\right\}$$
     Dann gilt $0\in\mathcal{G}$ und f\"ur $g,h\in\mathcal{G}$ ist $\max(g,h)\in\mathcal{G}$, denn f\"ur $A\in\mathcal{A}$ gilt
     $$\int_A \max(g,h)\ d\mu=\int\displaylimits_{A\cap\{g\geq h\}}g\ d\mu+\int\displaylimits_{A\cap\{g<h\}}h\ d\mu\overset{g,h,\in\mathcal{G}}{\leq}\nu(A\cap\{g\geq h\})+\nu(A\cap\{g<h\})=\nu(A).$$
      Setze nun $m:=\sup_{g\in\mathcal{G}}\int g\ d\mu<\nu(A)$ und wähle eine Folge $g_n\in\mathcal{G},n\geq1$, sodass
     $$\lim_{n\to\infty}\int g_n\ d\mu=m.$$
     Setze $f_n:=\max_{i\leq n}g_i$ und $f:=\sup_{n\geq1}g_n$, sodass 
     \begin{itemize}
     	\item $f_n\in\mathcal{G}$ für alle $n\geq1$,
     	\item $f_n(\omega)\nto{}{n\to\infty}f(\omega)$ für alle $\omega\in\Omega$, monoton nichtfallend in $n$,
     	\item $m\geq\int f_n\ d\mu\geq\int g_n\ d\mu\nto{}{n\to\infty}m$,
     \end{itemize}
     und daher
     $$\int f\ d\mu\overset{\text{MONK}}{=}\lim_{n\to\infty}\int f_n\ d\mu=m.$$
     Setze 
     $$\nu_s(A)=\nu(A)-\displaystyle\int_A f\ d\mu,\ A\in\mathcal{A}.$$ 
     Dann ist $\nu_s$ ein endliches Ma\ss{} (einfache Überlegung) und $\nu_s\perp\mu$. Angenommen nicht: Mit Lemma 12.6 gibt es $\eps>0,b\in\mathcal{A}$ mit $\mu(B)>0$, sodass $\forall A\in\mathcal{A}:\eps\cdot \mu(A\cap B)\leq\nu_s(A\cap B)$. Damit folgt f\"ur $A\in\mathcal{A}$
     \begin{align*}
         \int_A (f+\eps\cdot\ind{B})\ d\mu&=\int_A f\ d\mu+\eps\cdot\mu(A\cap B)\\
         &\leq\int_A f\ d\mu+\nu_s(A\cap B) \\
         &\leq\int_A f\ d\mu+\nu_s(A)=\nu(A)
     \end{align*}
     Damit folgt $f+\eps\cdot\ind{B}\in\mathcal{G}$ und 
     $$\sup_{g\in\mathcal{G}}\int g\ d\mu\geq\int(f+\eps\cdot\ind{B})\ d\mu=\int f\ d\mu\eps\cdot\mu(B)>\int f\ d\mu=\sup_{g\in\mathcal{G}}\int g\ d\mu$$
     ein Widerspruch.
     \newline\newline
     Es gilt per Definition
     $$\exists M\in\mathcal{A}:\mu(M)=0,\ \nu_s(M^c)=0.$$
     Da aber $\nu\ll\mu$ ist $\nu(M)=0$ und per Konstruktion
     $$0=\nu(M)0\geq\nu_s(M)\geq0$$
     und damit 
     $$\nu_s(\Omega)=\nu_s(M)+\nu(M^c)=0$$
     Also ist $\nu_s$ das Nullma\ss{} und $\nu(A)=\int_A f\ d\mu$, womit die erste Aussage folgt.\newline
     Sei nun $g$ eine weitere Dichte von $\nu$ bzgl. $\mu$ (also eine messbare Abbildung, die (7) erf\"ullt). Setze $A:=\{f>g\}$. Dann ist
     \begin{align*}
         0&=\nu(A)-\nu(A)\\
         &=\int_A f\ d\mu-\int_A g\ d\mu\\
         &=\int_A f-g\ d\mu \\
         &=\int\ind{\{f-g>0\}}(f-g)\ d\mu
     \end{align*}
     und da im Integral $f-g>0$, gilt $\ind{A}\overset{a.e.}{=}0$ und $\mu(A)=0$
     Ein \"ahnliches Argument folgt f\"ur $B:=\{f<g\}$ und die Aussage folgt schlie\ss{}lich mit der $\sigma$-Subadditivit\"at.\newline\newline
     2. Fall: ($\mu,\nu$ beide $\sigma$-endlich): W\"ahle $A_n\in\mathcal{A},n\geq1$ mit $A_1\subseteq\hdots\subseteq\bigcup_{n\geq1}A_n=\Omega$, sodass $\mu(A_n)<\infty$ f\"ur alle $n\geq1$. W\"ahle $B_n\in\mathcal{A}$ mit denselben Eigenschaften f\"ur $\nu$. Setze nun $C_n:=A_n\cap B_n$ f\"ur $n\geq1$, sodass $C_n\in\mathcal{A}$ und $C_1\subseteq\hdots\subseteq\bigcup_{n\geq1}C_n=\Omega$ und $\mu(C_n),\nu(C_n)<\infty$ f\"ur alle $n\geq1$. Disjunktisiere nun $C_n$ durch $M_1:=C_1$ und $M_{n+1}:=C_{n+1}\setminus C_n$ f\"ur alle $n\geq1$. Dann gilt $\Omega=\bigsqcup_{n\geq1}M_n$ und $\mu(M_n),\nu(M_n)<\infty$ f\"ur alle $n\geq1$. Definiere nun endlich Ma\ss{}e $\nu_n$ und $\mu_n$ wie folgt:
     \begin{gather*}
         \nu_n(A):=\nu(A\cap M_n) \\
         \mu_n(A):=\mu(A\cap M_n)
     \end{gather*}
     f\"ur alle $n\geq1$ und $A\in\mathcal{A}$. Laut Voraussetzung ist $\nu\ll\mu$ und daher auch $\nu_n\ll\mu_n$ f\"ur alle $n\geq1$. Mit dem 1. Fall hat also f\"ur jedes $n\geq1$ $\nu_n$ eine Dichte $f_n$ bzgl. $\mu_n$. Setze nun 
     $$f:=\sum_{n\geq1}f_n\cdot\ind{M_n}$$
     Dann ist $f\geq0$ und messbar und f\"ur $A\in\mathcal{A}$ gilt
     \begin{align*}
         \int_A f\ d\mu&=\int\sum_{n\geq1}f_n\cdot\ind{M_n}\cdot\ind{A}\ d\mu\\
         &\overset{\text{MONK}}{=}\sum_{n\geq1}\int f_n\cdot\ind{M_n}\cdot\ind{A}\ d\mu \\
         &=\sum_{n\geq1}\int_A f_n\cdot\ind{M_n}\ d\mu \\
         &=\sum_{n\geq1}\int_A f_n\ d\mu_n\\
         &=\sum_{n\geq1}\nu_n(A)\\
         &=\sum_{n\geq1}\nu(A\cap M_n)=\nu(A)
     \end{align*}
     Die Eindeutigkeit folgt wie im endlichen Fall. \qed
     
     \paragraph{12.8. Satz (Radon\textendash Nikodym Theorem f\"ur signierte Ma\ss{}e)} folgt.
     Sei $\varphi$ ein signiertes Maß und $\mu$ ein $\sigma$-endliches Maß, sodass $\varphi\ll\mu$.Dann hat $\varphi$ eine Dichte bez\"uglich $\mu$, i.e. es gibt ein $f:\Omega\to\R$ messbar, sodass
     \begin{equation*}
         \forall A\in\mathcal{A}: \varphi(A)=\int_A f \ d\mu
     \end{equation*}
     Wir schreiben $f=\dfrac{d\varphi}{d\mu}$. Falls insbesondere $g:\Omega\to\R$ eine weitere messbare Abbildung ist, die (7) erf\"ullt, dann gilt $f=g$ $\mu$-a.e.
     
     \paragraph{Beweis:}Betrachte die Hahn-Zerlegung $\Omega=\Omega_+\cup\Omega_-$ und die Jordan-Zerlegung $\varphi=\varphi_+-\varphi_-$. Dann sind $\varphi_+$ und $\varphi_-$ beide endliche Maße und $\varphi_+\ll\mu$, $\varphi_-\ll\mu$. Setze also 
     $$f_{(+)}:=\dfrac{d\varphi_+}{d\mu},\ f_{(-)}:=\dfrac{d\varphi_-}{d\mu},\ f:=f_{(+)}-f_{(-)}$$
     Die Messbarkeit von $f$ folgt sofort aus der Messbarkeit von $f_{(+)},f_{(-)}$ laut Satz 12.7. Weiters gilt $f_{(+)},f_{(-)}\in L^1(\mu)$ und damit auch $f\in L^1(\mu)$. Dass $f$ eine Dichte von $\varphi$ bez\"uglich $\mu$ ist folgt mit
     $$\varphi(A)=\varphi_+(A)-\varphi_-(A)=\int_A f_{(+)}\ d\mu-\int f_{(-)}\ d\mu=\int_A f\ d\mu$$
     Ist $g$ eine weitere Dichte von $\varphi$ bez\"uglich $\mu$, dann ist $g\cdot\ind{\Omega_+}$ eine Dichte von $\varphi_+$ bez\"uglich $\mu$ und mit Satz 12.7 gilt $g\cdot\ind{\Omega_+}\overset{a.e.}{=}f_{(+)}$. Dasselbe folgt f\"ur $g\cdot\ind{\Omega_-}$ und $f_{(-)}$. Da die Vereinigung zweier Nullmengen wieder eine Nullmenge ist folgt
     $$g\overset{a.e.}{=}f$$
     \qed
